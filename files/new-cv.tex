\documentclass[11pt, a4paper]{article}
\usepackage[utf8]{inputenc}
\usepackage[T1]{fontenc}
\usepackage{crimson} % Serif for narrative
\usepackage{inconsolata} % Monospace for technical data
\usepackage[margin=0.8in]{geometry}
\usepackage{hyperref}
\usepackage{titlesec}
\usepackage{enumitem}

% Style definitions
\titleformat{\section}{\large\bfseries\sffamily\uppercase}{}{0em}{}[\titlerule]
\titlespacing{\section}{0pt}{1.5ex}{1ex}
\setlist[itemize]{leftmargin=1.5em, nosep}

\begin{document}

% Header
\begin{center}
    {\Huge \textbf{Joan Alcaide-Núñez}} \\
    \vspace{0.5ex}
    {\small \texttt{Physics BSc Student @ LMU Munich}} \\
    \vspace{1ex}
    {\small \href{https://joanalnu.github.io}{joanalnu.github.io} | \href{mailto:your-email@provider.com}{Email}}
\end{center}

\section{Research Statement}
Physics candidate specializing in Multi-Messenger Cosmology. My work focuses on the integration of Gravitational Wave (GW) and Electromagnetic (EM) data through Bayesian frameworks to constrain cosmological parameters. I prioritize methodological transparency, original code development, and the pursuit of fundamental physical truths over metric-driven research.

\section{Research Experience}

\textbf{OAB-INAF | Brera-Merate Astronomical Observatory} \hfill \textit{2025} \\
\textit{Independent Research Stay -- Multi-Messenger Cosmology}
\begin{itemize}
    \item Developed a self-contained computational framework for GRB/GW parameter estimation from first principles, ensuring 100\% code sovereignty.
    \item Implemented Bayesian inference pipelines to fit Gamma-Ray Burst data, avoiding "black-box" software dependencies to ensure physical accuracy.
    \item \textbf{Advisors:} Dr. G. Ghirlanda, Dr. O. S. Salafia.
\end{itemize}

\vspace{1.5ex}

\textbf{ICE-CSIC | Institute of Space Sciences} \hfill \textit{2024} \\
\textit{Research Fellow -- Type Ia Supernovae}
\begin{itemize}
    \item Investigated the limits of SNIa distance moduli fits in the context of the Hubble Tension.
    \item Documented significant data dispersion in early-stage Hubble diagrams, utilizing the project as a case study in rigorous uncertainty quantification and skepticism toward over-smoothed results.
\end{itemize}

\vspace{1.5ex}

\textbf{CAPIBARA Collaboration} \hfill \textit{2023 -- Present} \\
\textit{Lead Coordinator \& Founder}
\begin{itemize}
    \item Coordinating a student-led initiative for high-energy instrumentation.
    \item Facilitating technical collaboration across institutions to develop low-cost X-ray/Gamma-ray satellite components.
\end{itemize}

\section{Technical Sovereignty \& Capabilities}

\textbf{Statistical \& Mathematical:} Bayesian Inference, Nested Sampling, MCMC, Likelihood Analysis, Error Propagation in Cosmological Ladders. \\
\textbf{Computational Physics:} Python (NumPy, SciPy, Astropy), C++, Git Version Control, Data Visualization for High-Energy Astrophysics. \\
\textbf{Instrumentation:} Raw data reduction, telescope operations, and instrumentation design for Gamma-ray detection.

\section{Education}
\textbf{Ludwig-Maximilians-Universität München} \hfill \textit{Oct 2025 -- Present} \\
Bachelor of Science in Physics \\
\textit{Current Focus:} Mastering the mathematical foundations of Theoretical Physics and General Relativity.

\vspace{1ex}

\textbf{Youth and Science Fellowship} \hfill \textit{2023 -- 2025} \\
Three-year selective research program focused on Astronomy and Cosmology. 

\section{Methodological Philosophy}
I maintain a strictly "Open Science" workflow. I believe that the role of a researcher is to be able to draft a theoretical framework or plan an experiment from a "blank page" (the "Desert Principle"). I decline to participate in trend-following or "Publish or Perish" cycles that compromise the depth of understanding for the sake of output volume. 

\vfill
\begin{center}
    \textit{\small Detailed project archives and source code available at \href{https://github.com/joanalnu}{github.com/joanalnu}}
\end{center}

\end{document}