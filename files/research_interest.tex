\documentclass[11pt]{article}
\usepackage[a4paper,margin=1in]{geometry}
\usepackage{amsmath,amssymb}
\usepackage{setspace}
\setstretch{1.1}

\begin{document}

\begin{center}
    {\Large \textbf{Statement of Research Interests}}\\[4pt]
    {\large Joan Alcaide-Núñez} \\[2pt]
    {\large \today}
\end{center}

\section*{Research Overview}
My research interests focus on \textbf{Multi-Messenger Cosmology} and the development of distance-ladder-independent probes. I am specifically interested in how joint Gravitational-Wave (GW) and Gamma-Ray Burst (GRB) detections can resolve the Hubble tension and probe dark energy dynamics. My goal is to move beyond standard $\Lambda$CDM constraints toward testing dynamical dark energy ($w_0w_a$CDM).

\section*{Foundational Trajectory}
My initiation into cosmological research began with a pre-undergraduate investigation into the local distance ladder. By recomputing the Hubble constant ($H_0$) using modern Cepheid and Type Ia Supernovae datasets, I gained early exposure to the systematic uncertainties inherent in traditional cosmic distance measurements. This work established my primary research motivation is to investigate independent, self-calibrating probes that bypass the intermediate steps of the distance ladder. Additionally, these projects allowed me to master computational skills.

\section*{Current Research: OAB-INAF & Multi-Messenger Synergy}
My most recent work, in collaboration with Dr. G. Ghirlanda and Dr. O. S. Salafia (OAB-INAF), transitioned these interests toward \textbf{high-redshift transients}. I investigated the $E_\text{peak}-E_\text{iso}$ (Amati) correlation in GRBs and evaluated how next-generation GW observatories like the \textbf{Einstein Telescope (ET)} will provide complementary standard siren measurements, with different degeneracies as electromagnetic probes. Current work focuses in applying Bayesian Inference to enhance the computational efficiency, and thus fit more parameters with the aim to fit beyond $\Lambda$CDM models. Alongside dark energy modelling, future work will include considering selection effects, potential redshift evolution, and decoupling intrinsic scatter from observational noise in the analysis.

\section*{Future Directions \& CAPIBARA}
In the long term, I aim to contribute to the mission architecture of the student-led initiative \textbf{CAPIBARA-COSMOS} program. My role focuses on the "Multi-Messenger Cosmology" initiative, advising the design and planning of the payload as well as working with simulations of synergies with GW observtories. I intend to pursue these questions through my Master’s and PhD, developing the computational tools necessary to phenomenologically study dark energy and cosmic expansion with multi-messenger events.

\end{document}