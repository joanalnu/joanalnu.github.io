\documentclass[11pt,a4paper]{article}

% ----------- Packages -----------
\usepackage[margin=1in]{geometry}
\usepackage{enumitem}
\usepackage{amsmath}
\usepackage{hyperref}
\hypersetup{
    colorlinks=true,
    linkcolor=blue,
    filecolor=magenta,      
    urlcolor=blue,
    pdftitle={Overleaf Example},
    pdfpagemode=FullScreen,
    }
\usepackage{titlesec}
\usepackage{parskip} % cleaner paragraph spacing

% ----------- Formatting -----------
\setlength{\parindent}{0pt}
\titleformat{\section}{\Large\bfseries}{}{0pt}{}
\titleformat{\subsection}{\bfseries}{}{0pt}{}

% Clean item spacing
\setlist[itemize]{topsep=2pt,itemsep=2pt,parsep=0pt}

% ----------- Document Begins -----------
\begin{document}

% ----------------- NAME + CONTACT -----------------
{\LARGE \textbf{Joan Alcaide-Núñez}} \\
\vspace{4pt}
\href{mailto:joan.alcaide@campus.lmu.de}{joan.alcaide@campus.lmu.de} \\
\href{https://joanalnu.github.io}{Website} \quad | \quad \href{https://www.linkedin.com/in/joan-alcaide-núñez}{LinkedIn} \quad | \quad \href{https://github.com/joanalnu}{GitHub} \\
Languages: English, German, Spanish, Catalan (basic Italian)

\vspace{1em}


% ----------------- PERSONAL STATEMENT -----------------
\section*{Personal Statement}

I am Joan Alcaide Núñez (he/him), a BSc Physics student at LMU Munich with a strong interest in observational and multi-messenger cosmology. My research focus lies in combining electromagnetic and gravitational-wave observations to constrain cosmological parameters and probe the nature of dark energy, with particular interest in the late-time evolution and fate of the Universe.

Between 2023 and 2025, I pursued a series of summer research projects centered on the cosmic distance ladder, recomputing the Hubble constant using Cepheids and then Type Ia Supernovae. Building on this foundation, I recently began to work with multi-messenger approaches, exploring how gamma-ray bursts (GRBs) and gravitational wave observations can jointly constrain cosmological models independently of, and complementary to, standard probes. These projects introduced me early to research-level data analysis and strengthened my mathematical, programming and statistical skills early beyond the high school curriculum.

After the Youth \& Science fellowship, I have continued to develop my research interests in dark energy and multi-messenger cosmology, extending my previous work and deepening my training. In particular, I have focused on applying Bayesian inference with Monte Carlo (extending from $\chi^2$ minimization), while learning about the theoretical foundations of dark energy models and alternatives to $\Lambda$CDM. I am now seeking opportunities to further develop these skills collaboratively within an active research environment.


% ----------------- RESEARCH EXPERIENCE -----------------
\section*{Research Experience}
\textbf{GRB cosmography with empirical energy relations} \hfill June - July 2025 \\
\textit{\href{https://brera.inaf.it/en}{Osservatorio Astronomico di Brera (OAB-INAF)}} \hfill \textit{5 weeks} \\
\textit{Supervisor: Dr. Giancarlo Ghirlanda, Dr. Om Sharan Salafia} \\
• Focus: Multi-Messenger Cosmology: combining Gamma-Ray Burst (GRB) data from $E_\text{\rm peak}-E_\text{\rm iso}$ relation (Amati relation) and a simulated population of binary neutron star (BNS) mergers detectable by Einstein Telescope \\
• Research stay at an research institute for 5 weeks during summer 2025.\\
• Skills: Gamma-Ray Bursts, relativity with photons, and relativistic plasma physics, working in logarithmic scales (leveraged), statistical methods ($\chi^2$, reduced $\chi^2$, partial differentiation, and more)\\
• Other: Engaging in journal clubs, night sky observations at the observatory, discussing research, collaboration with visiting and resident scientists (supervisors \& Ricardo Spinelli and Alberto Colombo)\\
• Supervised by Dr. Giancarlo Ghirlanda and Dr. Om Sharan Salafia (\href{https://sites.google.com/inaf.it/hea-merate/gamma-ray-bursts}{High-Energy Astrophysics Group})\\
• \href{https://github.com/joanalnu/OAB-INAF}{Repository}

\textbf{CAPIBARA Collaboration lead} \hfill 2024-present \\
\textit{\href{https://capibara3.github.io/}{CAPIBARA Collaboration}}\\
• CAPIBARA is a group of high-school and university students aiming to research the high-energy cosmos. It comprehends two missions (short-term pathfinder, long-term transient observatory) focusing on both engineering and scientific research.\\
• Established strategical partnerships with \href{https://pldspace.com/en}{PLD Space} \& \href{https://oba.space}{OBA Space} among others.\\
• Leading one of the research initiatives to use Gamma-Ray Burst data from the CAPIGX mission to constrain cosmological parameters and study the Universe at high redshift.\\
• CAPIGX also should be a useful resource for electromagnetic counterpart follow-ups in the multi-messenger era (2030s)\\
• Status: \href{https://capibara3.github.io/updates/y1_report_published.html}{Year 1 Collaboration Status Report} on July 2025

\textbf{Student Internship at German Space Agency} \hfill September 2024 \\
\textit{\href{https://www.dlr.de/de/eoc/ueber-uns/institut-fuer-methodik-der-fernerkundung}{Institute for Remote Sensing} - \href{https://dlr.de}{German Aerospace Center} (IMF-DLR)} \hfill 2 weeks

• First Week Activities: Worked with the Experimental Methods Department, learned about hyperspectral imaging, conducted outdoor field measurements, and participated in a drone-boat measuring experiments. Visited the \href{https://www.dlr.de/en/rb}{German Space Operation Center} (GSOC), where both satellite and human spaceflight are guided; the \href{https://www.dlr.de/en/eoc}{Earth Observation Center} (EOC).\\
• Second Week Activities: Worked the Photogrammetry and Image Processing Department, analyzed data using QGIS software and GDAL Python library, and studied temperature variations using Landsat 8 satellite thermal imaging. And visited the \href{https://www.dlr.de/en/gk}{Galileo Competence Center} (GK).

\textbf{Research Stay} \hfill June-July 2024 \\
\textit{\href{https://ice.csic.es}{Institute of Space Sciences (ICE-CSIC-IEEC)}} \hfill 2 months

• Focus: Utilizing Type Ia Supernovae in the infrared range as cosmological distance indicators. Analyzed data from ESO observatories (SOFI) and optical data from ATLAS and ZTF surveys to compute supernova distances, and fitted cosmological parameters such as the Hubble constant $H_0$ and dark energy density parameter $\Omega_\Lambda$. \\
• Skills: Supernova Ia, Python, Object-Oriented Programming (OOP), Paper Reading, Networking Aperture photometry, Data Analysis. \\
• Other: Engaging in journal clubs, seminars of visiting scientists, research discussions, and collaborating with scientists in astrophysics. \\
• Supervised by Dr. Lluís Galbany, principal researcher of the \href{https://www.ice.csic.es/about-us/organisation?view=article&id=208&catid=2}{Supernova and Stellar Transients Group}.

\textbf{Scientific Paper Writing} \hfill August - December 2023 \\
\textit{\href{https://www.fundaciocatalunya-lapedrera.com/en/youth-and-science}{Youth and Science Programme}}

• Writing of a scientific paper: Recomputing the Hubble constant. I became interested in cosmology and Hubble tension, addressing this problem by employed the same methods as 100 years ago with modern data from the NASA-IPAC NED Database and the Konkoly Observatory. \\
• Supervised by Dr. Ignasi Pérez-Ràfols and Dr. Laia Casamiquela. \\
• \href{https://github.com/joanalnu/article_hubble_jic}{Repository} \\
• Previously: Research stay in astronomy and astrophysics at MonNatura Pirineus, 10-body student group and 4 researchers; introduction to astronomical observations (16" Schmidt-Cassegrain telescope) and following data collection and analysis, astrophysics from star evolution to gravitational waves; practical demonstrations.


% ----------------- EDUCATION -----------------
\section*{Education}
\textbf{Ludwig-Maximilians-Universität (LMU) Munich} \hfill September 2025 -- Present \\
B.Sc. in Physics \\

\textbf{German School of Barcelona} \hfill September 2017 - May 2025 \\
Dual High School Diploma (Germany + Spain), GPA: 1.0/1.0


% ----------------- AWARDS -----------------
\section*{Awards}

• \textbf{Silver Honour and National Award Spain International Astronomy \& Astrophysics Competition (IAAC) 2025}, presented for the student with most points in the country, reaching the 7\% of $\sim$12300 students participating worldwide. \\
• \textbf{Physics Distinction for extraordinary grades by the German Physical Society (DPG)} \\
• \textbf{First Prize and Special Award in Scientific Photography in Jugend Forscht Nordrhein-Westfalen 2025}, forwarding to Final round \\
• \textbf{First Prize in Jugend Forscht Iberia 2025}, project title: "Cosmological distance measurements with Type Ia Supernovae" \\
• \textbf{Silver Honour Final Round of the International Astronomy \& Astrophysics Competition (IAAC) 2024} \\
• \textbf{Awarding of internship at German Space Agency (DLR) in Jugend Forscht Nordrhein-Westfalen 2024} \\
• \textbf{First Prize in Jugend Forscht Iberia 2024}, Spain + Portugal level, project title: "Recomputing the Hubble constant" \\
• \textbf{Bronze Honour in Final Round and National Award Spain in the International Astronomy \& Astrophysics Competition (IAAC) 2023}, presented for achieving the highest nation-wide score in the final round.


% ----------------- TECHNICAL SKILLS -----------------
\section*{Technical Skills}

\textbf{Programming Languages:} Python, C++, HTML/CSS/JavaScript \\
\textbf{Scientific Libraries:} numpy, scipy, pandas, matplotlib, astropy \\
\textbf{Tools:} Git, GitHub, \LaTeX{}, Markdown \\
\textbf{Methods:} data analysis, parameter estimation and model fitting, Bayesian inference, mathematical and cosmological functions \\

\textbf{Formal Training:}
\begin{itemize}[leftmargin=*]
    \item \textit{Gravity and Black Holes} — Perimeter Institute (GoPhysics!), 2025
    \item Mathematics, Statistics, and Data Science — Universitat Autònoma de Barcelona (UAB), 2022
    \item Algorithms and Programming (C++) — Universitat Politècnica de Catalunya (UPC), 2022
    \item League of Codes (C++) — Harbour Space, 2022--2023
\end{itemize}

\textbf{Currently Expanding:} emcee, GWTC \& gwcosmos, PyTorch


% ----------------- Selected Projects \& Outputs -----------------
\section*{Selected Projects \& Outputs}

\textbf{Gen10: Education Tool for Genomics} 
\href{https://joanalnu.github.io/gen10}{(website)} \hfill 2023--2025 \\
Developed a Python package and tutorials enabling students to explore basic genomics concepts through hands-on experimentation. 
\textit{Python, package development, testing, Jupyter notebooks}

\vspace{0.5em}

\textbf{28M: Live Local Elections Statistics}
\href{https://github.com/joanalnu/28M}{(GitHub)} \hfill Apr--May 2022 \\
Built a real-time data collection and visualization tool for local election results in collaboration with a regional radio station. 
\textit{Python, HTML/CSS, NumPy, xlwings}


% ----------------- TALKS -----------------
\section*{Outreach}

% \textbf{Symposium for Space Education Activities (SSEA, contributed)} \hfill April 2026 \\
% \textit{Technical University of Munich, TUM} \\
% Description.

\textbf{De l'aula a l'espai: joves, ions i fotons per entendre l'univers d'altes energies (Contributed)} \hfill March 2025 \\
\textit{CosmoXarxa, CosmoCaixa Barcelona's Science Museum} \\
CosmoXarxa is an initiative create by and for Explainers of the science museum (outreach volunteers). I presented the CAPIBARA project, its objectives and status, both technical, scientific and educative, inviting the public to join the efforts. \href{https://capibara3.github.io/publications/CosmoXarxa_presentation.pdf}{slides}

\textbf{Introduction to Astrophysics (Invited)} \hfill May/June 2024 \\
\textit{German School of Barcelona} \\
Double session (3h) about an naive introduction to astronomy and astrophysics for 11th graders, covered some cosmology (Big Bang theory), star evolution, galaxy morphology and massive bodies (neutron stars and black holes as extreme cases of star evolution). \href{https://joanalnu.github.io/documents/presentations/ClaseUniversoPu.pdf}{slides}

\textbf{L'Exploració Espacial (Invited)} \hfill October 2023 \\
\textit{Escola Canigó} \\
Presentation to pre-school students about human space travel and solar system exploration. Focused on hands-on learning with models and didactic, entertaining content. \href{https://joanalnu.github.io/documents/presentations/ExploracioEspacial.pdf}{slides}

\textbf{Explainers at CosmoCaixa Museum} \hfill 2022-2023 \\
Volunteered as Explainer at Barcelona's Science Museum explaining demonstrations to visitors on weekends. Participated in events such as Explainers during Winter, Explainers at Barcelona's City Science Fair and CosmoXarxa.


% ----------------- VOLUNTEERING & SOCIALS -----------------
\section*{Volunteering \& Socials}
\textbf{Cycling Without Age} \hfill 2024-2025 \\
Accompanying cycling "walks" around the village and enjoying sharing time with elderly.

\textbf{Music Band} \hfill 2021-2024 \\
Playing piano in different groups, school events, competitions, and for charity at local children's hospital. Continue playing


% ----------------- END -----------------
\end{document}
