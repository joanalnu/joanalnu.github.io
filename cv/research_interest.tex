\documentclass[11pt]{article}
\usepackage[a4paper,margin=1in]{geometry}
\usepackage{amsmath,amssymb}
\usepackage{setspace}
\setstretch{1.15}

\begin{document}

\begin{center}
    {\Large \textbf{Statement of Research Interests}}\\[4pt]
    {\large Joan Alcaide-Núñez}
\end{center}

\section*{Overview}

My interests lie at the intersection of \textbf{multi-messenger observations} and \textbf{theoretical models of dark energy} (cosmology). I am particularly motivated by the challenge posed by the \emph{Hubble tension} and by the possibility that gravitational waves (GW) and gamma-ray bursts (GRBs) may provide distance measurements that are independent of traditional cosmic distance ladder. Motivated by the enigma of dark energy and cosmic acceleration, my aim is to apply joint GW--GRB inference frameworks to constrain both standard cosmological parameters ($\Lambda$CDM), proposed extensions (such as $w_0w_a$CDM (CPL) parametrisation of dark energy), and alternative theoretical ideas including modified gravity phenomenology or cyclic scenarios (such as conformal cyclic cosmology (CCC)). 


\section*{Background and Motivation}

Before starting my undergraduate studies in physics, I have already come across with the question of how fast the Universe is expanding. During the last years of high school, I conducted projects on recomputing the Hubble--Lemaître constant ($H_0$) using modern Cepheid and Type Ia supernovae datasets. These projects introduced me to the distance ladder and the Hubble tension. They also were my first attempt in physics beyond school, giving me a wide background on coding and proceedings in data analysis and model fitting. This work led me naturally to seek \emph{distance-ladder-independent} probes to 'measure the universe'.

My latest research, conducted in collaboration with Dr. Giancarlo Ghirlanda and Dr. Om Sharan Salafia (OAB-INAF), has focused on the cosmological implications of GRBs through empirical relations such as the peak energy and equivalent isotropic energy ($E_\text{\rm peak}-E_\text{\rm iso}$ Amati correlation). Towards the end of my visiting stay, I dive into how next-generation GW observatories like Einstein Telescope (ET) will provide standard siren measurements unaffected by dust extinction or distance ladder calibration. This work has strengthened my interest in multi-messenger cosmology as a route to addressing the Hubble tension and probing the fundamental physics of dark energy.


\section*{Current Research Direction}

I am now especially interested in combining the strengths of GRB and GW observations into a unified statistical framework capable of constraining both conventional and alternative cosmological models at the same uncertainty levels as other methods.

My goals include:
\begin{itemize}
    \item \textbf{Move to Bayesian inference:} (instead of $\chi^2$ fitting) allowing for high dimensional parameter fitting, posterior likelihoods and parameter degeneracies analysis
    \item \textbf{Selection bias:} Malmquist bias, detector thresholds, detection probability dependence from $E_\text{\rm peak}, E_\text{\rm iso}, z$
    \item \textbf{Redshift evolution:} modify Amati relation to $\log E_\text{\rm peak} = a + b\log E_\text{\rm iso} + f(z)$
    \item \textbf{Systematics:} take systematics into account and separate it from observational uncertainty
    \item \textbf{Alternative theories:} fitting different models and comparing parameter ranges
\end{itemize}

In order to build this framework computational methods like Bayesian parameter inference, population and selections effects modeling and forward simulations (to mock multi-messenger catalogs) are central.

On the long-term, my goal is to use an increasing number of observations, and higher-z detections, to precisely try to constrain cosmological parameters and study the phenomenological signatures of dark energy and cosmological models.

\section*{Summary}

In future work, I hope to:
\begin{itemize}
    \item refine and expand joint GRB--GW cosmological inference frameworks
    \item investigate dark energy dynamics and modified gravity phenomenological signatures
    \item contribute to multi-wavelength and multi-messenger transient astronomy mission such as the CAPIBARA-COSMOS program aiming to support GW counterpart detections
    \item ultimately help develop new theoretical, observational, and computational tools capable of resolving the Hubble tension and advancing our understanding of dark energy
\end{itemize}

I see multi-messenger cosmology as a powerful route toward connecting dots and bringing insight into dark energy and the cosmic expansion. I hope to continue pursuing this intersection between multi-messenger observations and theoretical cosmological models throughout my Master's degree and PhD studies.

\end{document}
